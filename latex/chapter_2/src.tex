\chapter{Álgebras, Filtros e Morfismos}
    \epigraph{\justify
            A successful attempt to express 
            logical propositions by symbols, 
            the laws of whose combinations 
            should be founded upon the laws 
            of the mental processes which 
            they represent, would, so far, 
            be a step towards a philosophical 
            language.
        }{\textit{George Boole -- The Mathematical Analysis of Logic. p.5}}    
    \cls

    \paragraph{}
        O cálculo proposicional e as lógicas de primeira ordem 
        estão intimamente ligadas com estruturas que chamamos 
        de ``Álgebras''. Por esta razão, quando formos tratar 
        de Modelos a Valores Booleanos, convém compreender o quê
        é uma álgebra booleana, e quais propriedades exigimos 
        sobre esta a fim de fazer a construção.