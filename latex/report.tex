\documentclass[12pt]{report}
\usepackage[includeheadfoot,a4paper]{geometry}

\makeatletter
    \def\input@path{{../latex/}}
\makeatother

\input{_headers/all}

\title{{\bf Modelos a Valores Booleanos:}\\ \em Aplicações em Conjuntos e Álgebra}
% Não é boa prática usar \bf (https://tex.stackexchange.com/questions/41681/correct-way-to-bold-italicize-text)

\author{
    {\large  Prof.  Hugo Luiz Mariano}\\
                    José Goudet Alvim\\
    IME - USP
}
\date{2017-19}

\begin{document}
    \maketitle
    \justify
    \begin{center}{\bf Palavras-chave}\end{center}
        Reticulados; Álgebras de Heyting; Álgebras Booleanas;
        Teoria de Modelos; Valores booleanos de fórmulas;
        Independência; Forcing;
    \begin{center}{\bf Resumo}\end{center}
        Neste trabalho de iniciação científica, estudaremos
        a teoria de modelos a valores booleanos de $\sc ZF$,
        a fim de entender e produzir provas de consistência
        e independência na teoria de conjuntos. Para tanto,
        trataremos de álgebras de Boole; de Heyting; Filtros;
        Morfismos; Teoria de Modelos e Lógica.
    \cls

    \tableofcontents
    \cls

    \chapter{Lógica}
    \epigraph{\justify
        It seems to me now that mathematics
        is capable of an artistic excellence
        as great as that of any music
        \elide because it gives in absolute
        perfection that combination,
        characteristic of great art, of godlike
        freedom, with the sense of inevitable
        destiny; because, in fact, it constructs
        an ideal world where everything is
        perfect and yet true.
        }{\textit{Bertrand Russell.}}
    \cls
    \paragraph{}
        A forma mais básica de lógica que nos propomos
        a tratar é a ``lógica proposicional'', que
        veremos ser intimamente ligada com álgebras de % _a_ algebras de Heyting*
        Heyting. Ela consiste de uma coleção de letras
        proposicionais, que podem ser vistas como
        condições, e destas ``fórmulas atômicas'',
        obtemos fórmulas compostas através do uso de
        conectivos e operadores lógicos.
    \paragraph{}
        O estudo da lógica proposicional tem seu
        mérito, mas não será especialmente util para
        nós, afinal vamos tratar das álgebras de
        Heyting em profundidade. Mas, no entanto, % Não entendi. Estudar algebras de Heyting não _necessitaria_ o estudo de lógica proposicional?
        convém mostrar um pouco desta para contrastar
        com a {\em lógica de primeira ordem} que logo % Tbm tem o command \emph{<text>}.
        iremos tratar.
    \paragraph{}
        Grosseiramente, lógica proposicional se ocupa
        de sentenças como:
        $$(P\rar Q)\rar P\rar Q$$
        $$(P\land Q)\rar P$$
    \paragraph{}
        Pode-se ver como, associando ``valores de
        verdade'' às letras proposicionais, obtemos
        valores das fórmulas compostas. Mais que isso, % Porque obtemos valores de verdade? Mencionar que conectivos tbm tem interpretações.
        os teoremas da lógica proposicional são
        exatamente as tautologias. % Oras, o que é isso?
    \paragraph{}
        Avançando no nível de complexidade, no
        deparamos com a ``lógica de primeira ordem'',
        que trataremos agora com mais cuidado. % Não sei se precisa mencionar isso. O próximo subtítulo é exatamente esse.

    \section{Lógica de Primeira Ordem}
        \paragraph{}
            Nossa abordagem para lógica pode não ser
            a melhor para todos os propósitos, mas
            acreditamos que ela será util ao aproximar
            a estrutura de uma língua com uma
            {\em estrutura} de fato. % Não entendi..., mas ok.
        \paragraph{}
            Definimos uma língua formal como sendo
            as sentenças potenciais de uma gramática
            (finitária), isto é, dadas $\TUPLE{R_0}{\ldots}{R_n}$
            relações $\rho_0,\ldots,\rho_n$-árias,
            respectivamente; e uma coleção $\omega$ --- geralmente
            infinito potencial --- de variáveis
            $\SET{x_0}{x_1}{x_2}{\ldots}$, dizemos
            que uma sequência (finita) de símbolos $\sigma$ é
            da língua associada à assinatura
            exatamente quando vale alguma das
            condições abaixo:
        \begin{enumerate}[label=\alph*)]
            \item $\sigma\equiv R_k\POINT{x_{\alpha_0}}{\ldots}{x_{\alpha_{\rho_k}}}$, com $0\leq k\leq n$, e $x_{\alpha_i}$ todos em $\omega$.
            \item $\sigma\equiv \varphi\land\psi$, com $\varphi,\psi$ da língua.
            \item $\sigma\equiv \varphi\lor \psi$, com $\varphi,\psi$ da língua.
            \item $\sigma\equiv \neg\varphi$, com $\varphi$ da língua.
            \item $\sigma\equiv \forall x_i:\varphi$, com $x_i$ variável em $\omega$ e $\varphi$ da língua.
            \item $\sigma\equiv \exists x_i:\varphi$, com $x_i$ variável em $\omega$ e $\varphi$ da língua.
        \end{enumerate}
        \paragraph{}
            Como não iremos tratar de línguas com
            muitos símbolos, fixaremos uma coleção
            de variáveis $\SET{x_0}{x_1}{\ldots}$,
            para termos: dada uma assinatura $\Sigma= % Você não definiu “termo”. Além disso, não entendi o que tem a ver os poucos simbolos com fixar o conjunto de símbolos.
            \TUPLE{R_0}{\ldots}{R_n}$, associamos
            uma língua $\mathcal{L}_\Sigma$ como
            definido acima.

        \subsection{Satisfatibilidade}
        \paragraph{}
            Mas, afinal, qual é a interpretação
            destas formulas? Pois, o significado
            desta contrução vem exatamente da
            {\em definição} de interpretação, de
            significado. % O significado da construção vem da definição de significado? Entendi mais ou menos o que você quis fazer, mas não ficou muito bom.
        \paragraph{}
            Primeiro, observemos que para montar
            uma língua só usamos a quantidade
            de relações e a aridade de cada uma
            delas, $\rho_i$. Desta forma, vamos
            simplificar ainda mais e fixar
            $\SET{R_0}{R_1}{\ldots}$ símbolos
            canônicos para as relações, assim, % Dessa forma não fica impossível ter mais de um símbolo da mesma aridade?
            se tivermos
            $\Sigma=\TUPLE{\rho_0}{\ldots}{\rho_k}$,
            podemos definir $\mathcal{L}_\Sigma$
            como se espera: use os símbolos $R_i$
            para relações $\rho_i$-árias,
            respectivamente.
        \paragraph{}
            Segundo, definâmos ``estruturas'',
            pois se uma fórmula é verdade ela deve
            ser verdade em algum lugar (apesar que) % apesar _de_ que
            algumas coisas são verdade em todos os
            lugares. Para tanto, vamos definir uma
            ``valoração de variáveis'':

        \begin{definition}{Valoração de Variáveis}
                Dada uma classe $A$ de objetos, uma % Eu costumo colocar em negrito o termo que se está definindo. Não sei se seria uma boa pra você.
                valoração de variáveis das variáveis
                de uma língua $\mathcal{L}$ de
                primeira ordem com alguma assinatura,
                dizemos que $f$ é uma valoração de
                variáveis quando ela é uma realização
                dos símbolos das $\mathcal{L}$-variáveis
                em objetos de $A$. % Isso não significa muita coisa pra mim. Talvez adicionar um “_ie_, uma associação de variáveis a elementos do domínio.”
            \paragraph{}
                Adicionalmente, dados um objeto $o$
                de $A$, uma variável $x_j$ de
                $\mathcal{L}$ e uma valoração $f$,
                definimos:
                $$f[x_j\slash{o}](x_i) = \casedef{
                    o\text{, &se }i = j;\\
                    f(x_i)\text{, &se }i\not=j.
                }$$
        \end{definition}

        \paragraph{}
            Uma estrutura é uma coleção de informações:
            $$\mathfrak{A} = \TUPLE{A}{R_0}{\ldots}{R_n}$$
        \paragraph{}
            Onde $A$ é uma classe não vazia chamada ``domínio''
            e cada uma $R_i$ é uma relação definida
            sobre o mesmo. Dizemos que uma estrutura
            e uma língua são {\em compatíveis}
            exatamente quando a aridade de cada $R_i$
            é $\rho_i$. % Só tem uma relação de cada aridade, então? Btw, eu poderia ter uma lógica (+, *) e interpretá-la numa estrutura (R, #), associando + -> # e * -> #, não?

        \begin{definition}{Satisfatibilidade}
                Sejam $\varphi$ uma $\mathcal{L}$-fórmula;
                $f$ uma valoração de variáveis; e % Valoração em que domínio?
                $\mathfrak{A} = \TUPLE{A}{P_0}{\ldots}{P_n}$
                uma estrutura, e sejam $\mathcal{L}$ e
                $\mathfrak{A}$ compatíveis.
                $$ \mathfrak A\vDash_f\varphi$$
            \paragraph{}
                Que se lê  ``$\mathfrak{A}$ satisfaz
                $\varphi$'', ``$\mathfrak{A}$ crê que
                $\varphi$'' ou ``$\varphi$ vale em
                $\mathfrak{A}$'', É definido recursivamente:
            \begin{align*}
                \varphi\equiv R_i\POINT{x_0}{\ldots}{x_{\alpha_{\rho_i}}}
                    &\RAR \mathfrak A\vDash_f\varphi \bim P_i\POINT{f(x_0)}{\ldots}{f(x_{\alpha_{\rho_i}})}.\\
                \varphi\equiv \sigma\land\psi
                    &\RAR \mathfrak A\vDash_f\varphi \bim \mathfrak A\vDash_f\sigma \text{ e } \mathfrak A\vDash_f\psi.\\
                \varphi\equiv \sigma\lor \psi
                    &\RAR \mathfrak A\vDash_f\varphi \bim \mathfrak A\vDash_f\sigma \text{ ou } \mathfrak A\vDash_f\psi.\\
                \varphi\equiv \neg\sigma
                    &\RAR \mathfrak A\vDash_f\varphi \bim \text{ não }\mathfrak A\vDash_f\sigma.\\
                \varphi\equiv \sigma\rar\psi
                    &\RAR \mathfrak A\vDash_f\varphi \bim \mathfrak A\vDash_f\neg\sigma \text{ ou } \mathfrak A\vDash_f\psi.\\
                \varphi\equiv \forall x_i:\sigma
                    &\RAR \mathfrak A\vDash_f\varphi \bim \text{ para todo $a$ de $A$: } \mathfrak A\vDash_{f[x_i\slash{a}]}\varphi.\\
                \varphi\equiv \exists x_i:\sigma
                    &\RAR \mathfrak A\vDash_f\varphi \bim \text{ existe algum $a$ de $A$: } \mathfrak A\vDash_{f[x_i\slash{a}]}\varphi.
            \end{align*} % Não seria uma boa dar uma explicaçãozinha da definição? Tipo, “comentar o código”.
            \paragraph{}
                Existe uma questão delicada nessa quantificação
                que fazemos nas últimas duas clausulas: ela é
                necessáriamente de primeira ordem, desta forma, % Como assim é de primeira ordem?
                não definimos {\em nada}. A solução oferecida é % Como assim não definiram nada? Não entendi...
                tratar destas quantificações em um nível acima,
                em uma meta-teoria. % Esse parágrafo não fez muito sentido pra mim :/
        \end{definition}

        \paragraph{}
            Vamos tomar um parágrafo para explorar o significado
            da definição (admitidamente meta-teorética) de
            satisfatibilidade. Primeiro, vemos que se a fórmula
            sendo interpretada é um sentença\footnote{{\em ie.}
            todas as variáveis da fórmula estão quantificadas
            sobre por algum $\forall$ ou um $\exists$, em todos
            os escopos.}, então não importa quais valorações $f,
            f'$ que dermos, dada uma $\sigma$-sentença:
            $$ \mathfrak{A}\vDash_{f} \sigma \BIM \mathfrak{A}\vDash_{f'} \sigma $$
        \paragraph{}
            Segundo, se uma fórmula da língua é $\varphi\POINT{x}{y}{z}$\footnote{
            Onde $x,y,z$ são açucar sintático para alguns $x_i,
            x_j, x_k$.} onde estas vairáveis estão livres na
            fórmula, então se $\hat{x}, \hat{y}, \hat{z}$ forem
            objetos do domínio de uma estrutura compatível,
            ``$\varphi\POINT{\hat{x}}{\hat{y}}{\hat{z}}$'' ---
            que gostaríamos que significasse quão verdade é dizer % Esse “quão verdade” ficou legal, mas meio confuso. Talvez colocar um “quão verdade é dizer _que_”
            $\hat{x}, \hat{y}, \hat{z}$ estão relacionados como
            $\varphi$ predica --- se traduz simplesmente em:
            $f'=f[x\slash\hat x; y\slash\hat y; z\slash\hat z]$
            uma nova valoração de variáveis:
            $$ \mathfrak{A}\vDash_{f'} \varphi $$
        \paragraph{}
            Codifica exatamente isso. $\vDash_{f'}$ vale se e só
            se a estrutura dá uma semiose à $\varphi$ ``avaliada'' % WTF IS A SEMIOSE. Btw what is “a avaliada”?
            nos objetos $\hat{x},\hat{y},\hat{z}$. Assim,
            satisfatibilidade é a codificação dessa interpretação
            de uma coisa {\em sintática}, a fórmula, em uma coisa
            {\em semântica} relações entre objetos, quantificações
            sobre o domínio, etc.
        \paragraph{}
            Finalmente, --- dado um coleção de fórmulas $\Gamma$
            de uma língua compatível com $\mathfrak A$ estrutura e uma
            valoração $f$ ---, dizemos que:
            $$\mathfrak{A}\vDash_f\Gamma\BIM\text{ para toda }\varphi\text{ de }\Gamma\text{: }\mathfrak{A}\vDash_f\varphi$$
        \paragraph{}
            E que, se $\Gamma$ for coleção de sentenças, então dizemos:
            $$\mathfrak{A}\vDash\Gamma\BIM\text{ para toda }\varphi\text{ de }\Gamma\text{: }\mathfrak{A}\vDash_f\varphi$$
            onde $f$ é qualquer (fixa) valoração das variáveis, por exemplo: $f(x_i)=\hat{x}$\footnote{Pelo menos uma tal
            valoração existe pois os domínios de estruturas são não vazios.}

        \subsection{Provabilidade}
        \paragraph{}
            O interessante da satisfatibilidade como % “[...] da nossa noção de satisfabilidade é que [...]”
            definida antes é sua relação com a
            provabilidade. Para explora-la, devemos
            tratar de ``teorias''.
        \paragraph{}
            Agora, sendo um pouco colecionista,
            considere a coleção das fórmulas de
            de uma língua $\mathcal{L}_\Sigma$,
            com $\Sigma = \TUPLE{R_0}{\ldots}{R_n}$.
        \begin{definition}{Consequência Sintática}
                A ideia por trás da definição é que existem coisas
                que são verdadeiras independente de interpretação e
                localidade. Por exemplo, não importa se
                $\forall x_7\forall x_8:R_3(x_7,x_8)$ é verdade ou não
                em um uma dada estrutura (de assinatura compatível),
                sempre vale que: $[\forall x_7\forall x_8:R_3(x_7,x_8)]
                \rar[\exists x_7:\exists x_8:R_3(x_7,x_8)]$. Isto é,
                existem sentenças que não dependem de interpretações de
                variáveis nem da semântica das relações: suas verdades
                são {\em consequências sintáticas}.
            \paragraph{}
                Então vamos definir uma relação entre conjuntos de sentenças
                de uma língua e sentenças desta mesma língua a de consequência
                sintática.
            \paragraph{}
                Seja $\Gamma\subset\mathcal{L}$ conjunto de sentenças de uma
                língua. Seja também $\varphi$ uma sentença da mesma língua.
                Dizemos que $\varphi$ é consequência sintática de $\Gamma$:
                $$\Gamma\vdash\varphi\BIM(\mathfrak{A}\vDash\Gamma\RAR\mathfrak{A}\vDash\varphi)\footnote{
                    Onde $\mathfrak{A}$ é uma estrutura de assinatura compatível. Estamos, de certa forma,
                    com $\mathfrak{A}$ livre sobre estruturas, o que é assustador. Afinal, para teoria de
                    conjuntos, por exemplo, potencialmente estaríamos quantificando sobre classe próprias.
                    E, em teorias mais fortes, todo tipo de coisas informalizáveis. Já estamos, no entanto,
                    em uma meta-teoria generosa.\\ Pouco mais abaixo de onde estamos se agita um mar de metafísica, e questões
                    epistemo e ontológicas no qual não ousamos mergulhar.
                }$$
        \end{definition}
        \paragraph{}
            A definição se extende para conjuntos de sentenças simplemente dizendo
            $$\Gamma\vdash\Sigma\BIM\text{ para todo $\sigma$ de $\Sigma$ temos: } \Gamma\vdash\sigma$$

        \paragraph{}
            Esta é uma definição de fato bem estranha, definimos uma coisa ser sintáticamente
            consequente da outra em termos de interpretações, que são semânticas em
            natureza\footnote{
                É claro que é possível fazê-lo de maneira puramente sintática, que é útil
                para computadores e provadores automáticos de teoremas, mas cremos que é
                uma abordagem que oculta algo da beleza da relação inseparável entre $\vdash$
                e $\vDash$.\\
                Por outro lado, uma abordagem sintática {\em é} de fato mais precisa, pois não
                necessita quantificar sobre as estruturas compatíveis com a língua,
                {\em etc.}
            }. No entanto,
            ela é uma relação extremamente razoável: não há interpretação que tenha $\Gamma$
            sem ter $\varphi$. De alguma forma, dizer que ``Em todo lugar que $\Gamma$ vale,
            $\varphi$ vale'' nos garante que não importa o lugar, apenas $\Gamma$ e $\varphi$. % Gostei disso.

        \subsubsection{Teorias}
        \begin{definition}{Teoria}
                Dada uma língua, um subconjunto de sentenças $\mathcal{T}$ é dito uma teoria
                quando ele é não-vazio e $\vdash$-fechado. Isto é, se $\mathcal{T}\vdash\tau$
                então $\tau$ já estava em $\mathcal{T}$. Ou seja, é uma coleção de sentenças
                que contém todas as suas consequências semânticas.
        \end{definition}
        \paragraph{}
            \newcommand{\theoryof}[1]{\mathcal{T}_{#1}}
            Dada uma língua $\mathcal{L}$, e um conjunto de sentenças não-vazio desta que batizamos ``axiomas''
            $\mathcal{A}$, dizemos que um conjunto de sentenças $\mathcal{T}$ é a teoria de
            $\mathcal{A}$ exatamente quando {\em para toda $\tau$ de $\mathcal{T}$, temos
            $\mathcal{A}\vdash\tau$}, que podemos abreviar para $\theoryof{\mathcal{A}}$.
        \paragraph{}
            O fato de que, para um conjunto de axiomas conforme acima, $\theoryof{\mathcal{A}}$ é teoria
            verifica-se por:
        \begin{proof*}
            Suponha que $\theoryof{\mathcal{A}}\vdash\tau$.
            \\Então, $\mathfrak{A}\vDash\theoryof{\mathcal{A}}\RAR\mathfrak{A}\vDash\tau$.\\
            Mas $\mathfrak{A}\vDash\theoryof{\mathcal{A}}\RAR\mathfrak{A}\vDash\mathcal{A}$ pois
            $\mathcal{A}\subseteq\theoryof{\mathcal{A}}$.\\
            Por outro lado, $\mathfrak{A}\vDash\mathcal{A}\RAR\mathfrak{A}\vDash\theoryof{\mathcal{A}}$,
            pois, para cada $\sigma$ em $\theoryof{\mathcal{A}}$, $\mathcal{A}\vdash\sigma$.\\
            Assim, $\mathcal{A}\vdash\tau$, e portanto, $\tau$ está em $\theoryof{\mathcal{A}}$.
        \end{proof*}


\end{document}
